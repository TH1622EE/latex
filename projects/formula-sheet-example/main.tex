%%%%%%%%%%%%%%%%%%%%%%%%%%%%%%%%%%%%%%%%%%%%%%%%%%%%%%%%%%%%%%%%%%%%%%
% writeLaTeX Example: A quick guide to LaTeX
%
% Source: Dave Richeson (divisbyzero.com), Dickinson College
% 
% A one-size-fits-all LaTeX cheat sheet. Kept to two pages, so it 
% can be printed (double-sided) on one piece of paper
% 
% Feel free to distribute this example, but please keep the referral
% to divisbyzero.com
% 
%%%%%%%%%%%%%%%%%%%%%%%%%%%%%%%%%%%%%%%%%%%%%%%%%%%%%%%%%%%%%%%%%%%%%%
% How to use writeLaTeX: 
%
% You edit the source code here on the left, and the preview on the
% right shows you the result within a few seconds.
%
% Bookmark this page and share the URL with your co-authors. They can
% edit at the same time!
%
% You can upload figures, bibliographies, custom classes and
% styles using the files menu.
%
% If you're new to LaTeX, the wikibook is a great place to start:
% http://en.wikibooks.org/wiki/LaTeX
%
%%%%%%%%%%%%%%%%%%%%%%%%%%%%%%%%%%%%%%%%%%%%%%%%%%%%%%%%%%%%%%%%%%%%%%

\documentclass[a4paper]{article}
\usepackage{amssymb,amsmath,amsthm,amsfonts}
\usepackage{multicol,multirow}
\usepackage{calc}
\usepackage{ifthen}
\usepackage[landscape]{geometry}
\usepackage[colorlinks=true,citecolor=blue,linkcolor=blue]{hyperref}


\ifthenelse{\lengthtest { \paperwidth = 11in}}
    { \geometry{top=.5in,left=.5in,right=.5in,bottom=.5in} }
	{\ifthenelse{ \lengthtest{ \paperwidth = 297mm}}
		{\geometry{top=1cm,left=1cm,right=1cm,bottom=1cm} }
		{\geometry{top=1cm,left=1cm,right=1cm,bottom=1cm} }
	}

\pagestyle{empty}
\makeatletter
\renewcommand{\section}{\@startsection{section}{1}{0mm}%
                                {-1ex plus -.5ex minus -.2ex}%
                                {0.5ex plus .2ex}%x
                                {\normalfont\large\bfseries}}
\renewcommand{\subsection}{\@startsection{subsection}{2}{0mm}%
                                {-1explus -.5ex minus -.2ex}%
                                {0.5ex plus .2ex}%
                                {\normalfont\normalsize\bfseries}}
\renewcommand{\subsubsection}{\@startsection{subsubsection}{3}{0mm}%
                                {-1ex plus -.5ex minus -.2ex}%
                                {1ex plus .2ex}%
                                {\normalfont\small\bfseries}}
\makeatother
\setcounter{secnumdepth}{0}
\setlength{\parindent}{0pt}
\setlength{\parskip}{0pt plus 0.5ex}
% -----------------------------------------------------------------------

\title{Communications Formula Sheet}

\begin{document}

\raggedright
\footnotesize

\begin{center}
     \Large{\textbf{Signals and Systems Formula Sheet}} \\
\end{center}
\begin{multicols}{3}
\setlength{\premulticols}{1pt}
\setlength{\postmulticols}{1pt}
\setlength{\multicolsep}{1pt}
\setlength{\columnsep}{2pt}

\subsection{Trigonometric Fourier Series}
\begin{align*}
    x(t) &= a_0 + \sum_{n=1}^{\infty} a_n \cos{n w_0 t} + \sum_{n=1}^{\infty} b_n \sin{n w_0 t} \\
    where,\\
    a_0 &= \dfrac{1}{T} \int_{-T/2}^{T/2} x(t) \,dt \\
    a_n &= \dfrac{2}{T} \int_{-T/2}^{T/2} x(t) \cos{n w_0 t} \,dt \\
    b_n &= \dfrac{2}{T} \int_{-T/2}^{T/2} x(t) \sin{n w_0 t} \,dt \\
\end{align*}

\subsection{Periodic Signals}
\begin{align*}
    x(t + T) = x(t) \hspace*{0.1cm} \text{for all $t$} \hspace*{0.1cm} \textit{,} \hspace*{0.1cm} -\infty < t < \infty
\end{align*}

\subsection{Fundamental Frequency}
\begin{align*}
    f = \dfrac{1}{T_0}
\end{align*}

\subsection{Fundamental Angular Frequency}
\begin{align*}
   \omega_0 = 2 \pi f = \dfrac{2 \pi}{T_0}
\end{align*}

\subsection{Fundamental Period}
\begin{align*}
    T_0 = \dfrac{2 \pi}{\omega_0} = \dfrac{1}{f_0}
\end{align*}

\section{TDPT}
\begin{align*}
    H = H_0 + V(t) \\
    \text{Eigenstate of $H_0$: } |n\rangle, E_n\\
    \text{transition: } |i\rangle \to | f\rangle \\
    V_{f i}(t) = \langle f | V(t) | i \rangle \\
    \omega_{f i} = (E_f-E_i)/\hbar \\
    c_f(T) = \frac{-i}{\hbar} \int_0^T V_{f i}(t) e^{-i \omega_{f i} t} dt 
\end{align*}

\subsection{Constant Perturbation}
\begin{align*}
    V(t) = \begin{cases}
    V, & 0 \leq t \leq T \\
    0, & \text{otherwise}
    \end{cases} \\
    V_{fi}(t)= constant \\
    P_{i\to f}(t)= |c_f(t)|^2 = 4 \frac{|V_{fi}|^2}{\hbar^2}  \frac{\sin^2(\omega_{fi})t/2}{\omega_{fi}^2} \\
    \omega_{fi} \to 0 \quad \text{(degenerate states):} \\
    |c_f(t)|^2 = \frac{|V_{fi}|^2}{\hbar^2} t^2
\end{align*}

\subsection{Absorption}
\begin{align*}
    V(t) = V \sin(\omega t) \\
    P_{i\to f}(t) = \frac{|V_{fi}|^2}{\hbar^2}  \frac{\sin^2((\omega_{fi}-\omega)t/2)}{(\omega_{fi}-\omega)^2}
\end{align*}

\subsection{Simulated Emission}
\begin{align*}
    E_i > E_f,\quad \omega_{fi}<0 \\
    P_{i\to f}(t) = \frac{|V_{fi}|^2}{\hbar^2}  \frac{\sin^2((\omega_{fi}+\omega)t/2)}{(\omega_{fi}+\omega)^2}
\end{align*}

\subsection{Fermi Golden Rule}
\begin{align*}
    E_i \to E_f\text{  (continuous states)} \\
    P_{i\to f} = 
\frac{2\pi}{\hbar} |\langle f | V| i\rangle |^2 \rho(E_f) t \\
\end{align*}

\subsection{Selection Rule}
For spherical symmetric potential: 
\begin{align*}
    \langle n',l',m'| \vec{r} | n,l,m \rangle &\neq 0 \text{  when:} \\
    \Delta l &= \pm 1 \text{  and:}\\
    \Delta l &= \pm 1 \text{ or } 0
\end{align*}


\section{Scattering}
\begin{align*}
    \psi(r,\theta) &= e^{ikz} + f(\theta) \frac{e^{ikr}}{r}, \text{  for large r} \\
    k &= \frac{\sqrt{2mE}}\hbar \\
    \frac{d \sigma}{d \Omega} &= |f(\theta)|^2 \\
    \sigma &= \int d\omega \frac{d \sigma}{d \Omega}
\end{align*}

\subsection{Born Approximation}
\begin{align*}
    f(\theta) = -\frac{m}{2\pi \hbar^2} \int  V(\vec{r}) e^{i(\vec{k}'-\vec{k})\cdot \vec{r}} d^3\vec{r}
    \intertext{Low Energy:}
    f(\theta) = -\frac{m}{2\pi \hbar^2} \int  V(\vec{r}) d^3\vec{r} 
    \intertext{Spherical symmetric:}
    f(\theta) = -\frac{2m}{\hbar^2 \kappa} \int_0^\infty r V(r) \sin(\kappa r) dr \\
    \kappa = 2k\sin(\theta/2)
\end{align*}

\subsubsection{Yukawa Potential}
\begin{align*}
    V(r) = V_0 \frac{e^{-r/R}}{r} \\
    f(\theta) = -\frac{2mV_0 R^2}{\hbar^2} \frac{1}{1+4k^2R^2\sin^2(\theta/2)} \\
    \sigma = (\frac{2mV_0R^2}{\hbar^2})^2 \frac{4\pi}{1+4k^2R^2}
\end{align*}

\subsubsection{Rutherford Scattering}
Let $V_0 = q_1q_2/4\pi \epsilon_0$, $R=\infty$:
\begin{align*}
    f(\theta) = -\frac{2mq_1q_2}{4\pi\epsilon_0\hbar^2\kappa^2}
\end{align*}

\subsection{Partial Waves}
\begin{align*}
    f(\theta) &= \frac1k \sum_{i=0}^\infty (2l+1)e^{i \delta_l} \sin(\delta_l) P_l(\cos(\theta)) \\
    \sigma &= \frac{4\pi}{k^2}\sum_{l=0}^\infty (2l+1) \sin^2(\delta_l)
\end{align*}

\subsubsection{Optical Theorem}
\begin{equation*}
    Im[f(0)] = \frac{k \sigma}{4\pi}
\end{equation*}

\subsubsection{Hard Ball}
\begin{align*}
    \delta_l = \tan^{-1}(\frac{j_l(ka)}{\eta_l(ka)}) \\
     ka << 1 \to \sigma = 4\pi a^2
\end{align*}

\section{Useful Models}

\subsection{Density of States}
\begin{align*}
    E &= \hbar^2 k^2 /2m \\
    dN &= \frac{L^3}{(2\pi)^3} d^3k = \frac{L^3}{(2\pi)^3} d\Omega dk \\
    dN &= \frac{L^3}{(2\pi)^3} 4\pi \frac{m}{\hbar^2 k} dE \\
    \rho(E) &= \frac{dN}{dE} = \frac{L^3}{2\pi^2} \frac{mk}{\hbar^2}
\end{align*}

\subsection{infinite square well}
\begin{align*}
    H(x) = \frac{p^2}{2m} + \begin{cases}
    0, & 0\leq x\leq a \\
    \infty, & \text{otherwise}
    \end{cases} \\
    E_n = \frac1{2m} (\frac{n \pi \hbar}{a})^2 \\
    \psi_n = \sqrt{\frac{2}{a}} \sin(\frac{n\pi x}{a}) e^{-iE_nt/\hbar}
\end{align*}

\subsection{Harmonic Oscillator}
\begin{align*}
    H(x) &= \frac{p^2}{2m} + \frac12 m\omega^2x^2 \\
    E_n &= (n+1/2)\hbar \omega \\
    \psi_n(x) &= \frac1{\sqrt{2^n n!}} (\frac{m\omega}{\pi \hbar})^{1/4} e^{-\zeta^2/2} H_n(\zeta) \\
    \zeta &= \sqrt{\frac{m\omega}{\hbar}x}
\end{align*}

\subsection{Virial Theorem}
\begin{align*}
    2 \langle T \rangle = \langle \vec{r} \cdot \nabla V \rangle 
    \quad \text{(3D)}\\
    2 \langle T \rangle = \langle x  \frac{dV}{dx} \rangle \quad \text{(1D)}\\
    2 \langle T \rangle = n \langle V \rangle \quad (V\propto r^n) \\
    \langle T\rangle = -E_n, \quad \langle V \rangle = 2 E_n \quad \text{(hydrogen)}\\
    \langle T\rangle = \langle V \rangle = E_n/2 \quad \text{(harmonic oscillator)}\\
\end{align*}

\section{Math}

\subsection{Legendre Polynomials}
Domain: $(-1,1)$ \\
Even, Odd, Even, Odd ...
\begin{align*}
    P_0(x) &= 1 \\
    P_1(x) &= x \\
    P_2(x) &= \frac12 (3x^2-1) \\
    P_3(x) &= \frac12 (5x^3 - 3x) 
\end{align*}

\subsection{Hankel Functions}
Solution to Radial Shrodinger Equation:
\begin{align*}
    -\frac{\hbar^2}{2m} \frac{1}{r^2} \frac{\partial}{\partial r} (r^2 R_{El}) + [V(r) + \frac{\hbar^2 l (l+1)}{2m r^2}]R_{El} = E R_{El} \\
    V = 0 \to R_{El} = j_l(kr) \\
    V \neq 0 \to R_{El} = j_l(kr+\delta_l) \\
    r \to \infty \Rightarrow R_{El} = \frac{\sin(kr-l\pi/2+\delta_l(E))}{kr}
\end{align*}
When $kr >> 1$ 
\begin{align*}
    j_l(kr) &\to \frac{\sin{kr-l\pi/2}}{kr} \\
    \eta_l(kr) &\to \frac{-\cos{kr-l\pi/2}}{kr} \\
    h_l(kr) &\to \frac{e^{i(kr-l\pi/2)}}{ikr} \\
    h_l^*(kr) &\to \frac{e^{-i(kr-l\pi/2)}}{-ikr} \\
    j_l(kr) &= \frac{1}{2} (h_l(kr)+h^*(kr))
\end{align*}

\subsection{Hermite  Polynomials}
Domain: $(-\infty,\infty)$ \\
Even, Odd, Even, Odd ...
\begin{align*}
    H_0(x) &= 1 \\
    H_1(x) &= 2x \\
    H_2(x) &= 4x^2-2 \\
    H_3(x) &= 8x^3-12x 
\end{align*}

\subsection{Spherical Harmonics}
\begin{align*}
    |l,m \rangle &= Y_l^m(\theta, \phi) \\
    Y_0^0(\theta,\phi) &= \frac12 \frac1{\sqrt{\pi}} \\
    Y_1^0(\theta,\phi) &= \frac12 \sqrt{\frac3\pi} \cos{\theta} \\
    Y_1^{-1}(\theta,\phi) &= \frac12 \sqrt{\frac{3}{2\pi}} \sin{\theta} e^{-i \phi} \\
    Y_1^{-1}(\theta,\phi) &= -\frac12 \sqrt{\frac{3}{2\pi}} \sin{\theta} e^{i \phi} 
\end{align*}

\subsection{Green's Function}
For a Linear Operator $\hat{D}_x$
\begin{align*}
    \text{Homogeneous solution:  }\hat{D}_x \psi_0(x) = 0 \\
    \text{Hard Problem:  }\hat{D}_x \psi(x) = f(x) \\
    \text{Simple Problem:  }\hat{D}_x G(x,x') = \delta(x-x') \\
    \psi(x) = \psi_0(x) + \int_{\text{f Domain}} G(x,x') f(x') dx'
\end{align*}

\subsection{Some Integrals}
\begin{align*}
    \Gamma(n+1) &=  n! \\
    \Gamma(z+1) &= z \Gamma(z) \\
    \int_0^\infty x^n e^{-ax} dx &= \frac{n!}{a^{n+1}} \\
    \int_0^\infty e^{-ax^b} dx &= a^{-1/b} \Gamma(1/b+1) \\
    \int_0^\infty e^{-ax} \sin{bx} dx &= \frac{b}{a^2+b^2} \\
    \int_0^\infty e^{-ax} \cos{bx} dx &= \frac{a}{a^2+b^2} \\
    \int_{-\infty}^\infty e^{-ax^2+bx} dx &= \sqrt{\frac{\pi}{a}} e^{\frac{b^2}{4a}} \\
    \int_0^\infty e^{-ax^2}x^n dx &= I_n(a)\\
    I_0=\frac12 \sqrt{\frac{\pi}{a}}, I_1&=\frac{1}{2a}, I_2=\frac1{4a} \sqrt{\frac{\pi}{a}}, I_3=\frac{1}{2a^2}
\end{align*}


\end{multicols}

\end{document}
